\title{Lab assignment 1 \\ \small{Datamining}}
\author{Chiel ten Brinke 3677133}
\documentclass[12pt]{article}
\usepackage{amssymb,amsmath,amsthm,enumerate,graphicx,float,lmodern}

\newtheorem{theorem}{Theorem}[section]
\newtheorem{lemma}[theorem]{Lemma}
\newtheorem{proposition}[theorem]{Proposition}
\newtheorem{corollary}[theorem]{Corollary}

\theoremstyle{definition}
\newtheorem{definition}[theorem]{Definition}
\newtheorem{axiom}[theorem]{Axiom}
\newtheorem{example}[theorem]{Example}
\newtheorem{remark}[theorem]{Remark}

\newcommand{\set}[2]{\left\lbrace#1 \, \middle|\, #2 \right\rbrace}

\begin{document}
\maketitle

\section{Dataset}
\label{sec:dataset}

The dataset that has been used for this assignment is called Covertype Data Set and has been
taken from the UCI machine learning
repository\footnote{http://archive.ics.uci.edu/ml/datasets/Covertype}.
There are relatively many attributes and the dataset appears to be very predictable.
To make the results a bit more interesting, we shall disregard some attributes.
Besides the fact that there are many attributes, there are also a lot of data instances,
namely 581012.
Because this large amount of data leads to inconvenient running times, and because of the
predictability of the dataset, we shrinked the dataset in a randomized manner,
down to a size that is more suitable for this assignment.


\end{document}
