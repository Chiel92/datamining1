\title{Lab Assignment 1 \\ \small{Datamining}}
\author{Chiel ten Brinke 3677133}
\documentclass[12pt]{article}
\usepackage{amssymb,amsmath,amsthm,enumerate,graphicx,float,lmodern}

\newtheorem{theorem}{Theorem}[section]
\newtheorem{lemma}[theorem]{Lemma}
\newtheorem{proposition}[theorem]{Proposition}
\newtheorem{corollary}[theorem]{Corollary}

\theoremstyle{definition}
\newtheorem{definition}[theorem]{Definition}
\newtheorem{axiom}[theorem]{Axiom}
\newtheorem{example}[theorem]{Example}
\newtheorem{remark}[theorem]{Remark}

\newcommand{\set}[2]{\left\lbrace#1 \, \middle|\, #2 \right\rbrace}

\begin{document}
\maketitle

\section{Data Set}
\label{sec:dataset}

The data set that has been used for this assignment is called Covertype Data Set and has been
taken from the UCI machine learning
repository\footnote{http://archive.ics.uci.edu/ml/data sets/Covertype}.
There are relatively many attributes and the data set appears to be very predictable.
To make the results less trivial, we shall disregard some attributes.
Besides the fact that there are many attributes, there are also a lot of data instances,
namely $581012$.
Because this large amount of data leads to inconvenient running times, and because of the
predictability of the data set, we shrinked the data set in a randomized manner,
down to a size that is more suitable for this assignment.
The resulting data set contains $10000$ instances.

The training set and test set are sampled randomly from the data set, such that
they are complementary and have ratio 7:3 as required.

As classlabel column 13 is used, and as attributes columns 1 till 10 are used.
This is because column 13 is the binary label where the number 0's and 1's are
most equally distributed.
Most of the other binary labels are very unequally distributed, giving trivial results.

\section{Results}
\label{sec:results}


\end{document}
